\documentclass{article}

\usepackage{amsmath,amssymb,amsthm}
%\usepackage{unicode-math}
\usepackage{url}
\usepackage{fullpage}

\usepackage{subcaption}
\usepackage{cedilleverbatim}
\DeclareUnicodeCharacter{03BC}{\ensuremath{\mu}}
\DeclareUnicodeCharacter{21A6}{\ensuremath{\mapsto}}
\DeclareUnicodeCharacter{25CF}{\ensuremath{\medbullet}}

\begin{document}

\title{The Cedilleum Language Specification \\ \large Syntax, Typing, Reduction,
  and Elaboration }

\author{Christopher Jenkins}

\maketitle

\section{Syntax}

\begin{figure}[h]
  \[
    \begin{array}{llll}
      id & &
      & \textnormal{identifiers for definitions}
      \\ u & &
      & \textnormal{term variables}
      \\ X & &
      & \textnormal{type variables}
      \\ k & &
      & \textnormal{kind variables}
      \\ x & ::= & id\ |\ u\ |\ X\
      & \textnormal{non-kind variables}
    \end{array}
  \]
  \caption{Identifiers}
\end{figure}

\begin{figure}[h]
  \[
    \begin{array}{llll}
      uterms
      & ::= & u
      \\ & & \textbf{λ}\ u \textbf{.}\ uterm
      \\ & & uterm\ uterm
    \end{array}
  \]
  \caption{Untyped terms}
\end{figure}

\begin{figure}[h]
  \[
    \begin{array}{llll}
      % module stuff
      \\ mod
      & ::= & \textbf{module}\ id\ \textbf{.}\ imprt^*\ cmd^*\
      & \textnormal{module declarations}
      \\ imprt
      & ::= & \textbf{import}\ id\ \textbf{.}
      & \textnormal{module imports}
      \\ cmd
      & ::= & defTermOrType
      & \textnormal{definitions}
      \\ & & defDataType
      \\ & & defKind
      % definitions
      \\ 
      \\ defTermOrType
      & ::= & id\ checkType^?\ \textbf{=}\ term\ \textbf{.}
      & \textnormal{term definition}
      \\ & & id\ \textbf{:}\ kind\ \textbf{=}\ type\ \textbf{.}
      & \textnormal{type definition}
      \\ defDataType
      & ::= & \textbf{data}\ id\ param^*\ \textbf{:}\ kind\ \textbf{=}\
              constr^*\ \textbf{.}
      & \textnormal{datatype definitions}
      \\ defKind
      & ::= & k\ \textbf{=}\ kind
      & \text{kind definition}
      % auxilliary categories for definitions
      \\ 
      \\ checkType
      & ::= & \textbf{:}\ type
      & \textnormal{annotation for term definition}
      \\ param
      & ::= & \textbf{(}x\ \textbf{:}\ typeOrKind \textbf{)}
      \\ typeOrKind
      & ::= & type
      \\ & & kind
      \\ constr
      & ::= & \textbf{\textbar}\ id\ \textbf{:}\ type
    \end{array}
  \]
  \caption{Modules and definitions}
\end{figure}

\begin{figure}
  \[
    \begin{array}{llll}
      \\ kind
      \\ & ::= & \textbf{Π}\ x\ \textbf{:}\ typeOrKind\ \textbf{.}\ kind
         & \textnormal{explicit product}
      \\ & & typeOrKind\ \textbf{➔}\ kind
         & \textnormal{kind arrow}
      \\ & & \textbf{★}
      \\ & & \textbf{(}kind\textbf{)}
      \\ type
      \\ & ::= & \textbf{Π}\ x\ \textbf{:}\ type\ \textbf{.}\ type
         & \textnormal{explicit product}
      \\ & &  \textbf{∀}\ x\ \textbf{:}\ typeOrKind\ \textbf{.}\ type
         & \textnormal{implicit product}
      \\ & &  \textbf{λ}\ x\ \textbf{:}\ typeOrKind\ \textbf{.}\ type
         & \textnormal{type-level function}
      \\ & & type\ \textbf{➾}\ type
         & \textnormal{arrow with erased domain}
      \\ & & type\ \textbf{➔}\ type
         & \textnormal{normal arrow type}
      \\ & & type\ \textbf{·}\ type
         & \text{application to another type}
      \\ & & type\ term
         & \text{application to a term}
      \\ & & \textbf{\{}\ uterm\ ≃\ uterm \textbf{\}}
         & \textnormal{untyped equality}
      \\ & & \textbf{(}type\textbf{)}
      \\ & & X
         & \text{type variable}
      \\ & & \bullet
         & \text{hole for incomplete types}
    \end{array}
  \]
  \caption{Kinds and types}
\end{figure}

\begin{figure}[h]
  \[
    \begin{array}{llll}
      \\ term
      & ::= & \textbf{λ}\ x\ class^?\ \textbf{.}\ term
      & \textnormal{normal abstraction}
      \\ & & \textbf{Λ}\ x\ class^?\ \textbf{.}\ term
      & \textnormal{erased abstraction}
      \\ & & \textbf{[}\ defTermOrType\ \textbf{]}\ \textbf{-}\ term
      & \textnormal{let}
      \\ & & \textbf{ρ}\ term\ \textbf{-}\ term
      & \text{equality elimination by rewriting}
      \\ & & \textbf{φ}\ term\ \textbf{-}\ term\ \textbf{\{} term \textbf{\}}
      & \text{type cast}
      \\ & & \textbf{χ}\ type^?\ \textbf{-}\ term
      & \text{check a term against a type}
      \\ & & \textbf{δ}\ \textbf{-}\ term
      & \text{ex falso quodlibet}
      \\ & & \textbf{θ}\ term\ term
      & \text{elimination with a motive}
      \\ & & term\ term
      & \text{applications}
      \\ & & term\ \textbf{-}\ term
      & \text{application to an erased term}
      \\ & & term\ \textbf{·}\ type
      & \text{application to a type}
      % \\ & & term\ arg^*
      % & \textnormal{applications}
      \\ & & \textbf{β}\ \textbf{\{} term \textbf{\}}
      & \textnormal{reflexivity of equality}
      \\ & & \textbf{ς}\ term
      & \textnormal{symmetry of equality}
      \\ & & \textbf{μ}\ term\ motive^?\ \textbf{\{}\ case^*\ \textbf{\}}
      & \textnormal{pattern match and fixpoint}
      \\ & & u
      & \text{term variable}
      \\ & & \textbf{(}term\textbf{)}
      \\ & & \bullet
      & \text{hole for incomplete term}
      \\
      \\ vararg
      & ::= & u
      & \text{normal constructor argument}
      \\ & & \textbf{-}\ u
      & \text{erased constructor argument}
      \\ & & \textbf{·}\ X
      & \text{type constructor argument}
      \\ class
      & ::= & \textbf{:}\ typeOrKind
      \\ motive
      & ::= & \textbf{@}\ type
      & \textnormal{motive for induction}
      \\ case
      & ::= & \textbf{\textbar}\ id\ arg^*\ \textbf{↦}\ term
      & \text{pattern-matching cases}
    \end{array}
  \]
  \caption{Annotated Terms}
\end{figure}
\end{document}